\chapter{Conclusions and Future Work}
\label{chap:conclusions}

We have presented a fully automatic optimizing compiler for Futhark, a
pure functional array language.  Futhark is a simple language that
supports just a few core concepts, yet is more expressive than prior
parallel languages of similar performance, in particular by supporting
nested parallelism.  While more flexible parallel languages exist
(notably NESL), these have not yet been shown to obtain good GPU
performance in practise.

We have demonstrated a combination of imperative and functional
concepts by supporting in-place updates with safety guaranteed by
uniqueness types rather than complicated index-based analysis.  We
support not just efficient top-level imperative code with in-place
updates, but also sequential code nested inside parallel constructs,
all without violating the functional properties on which we depend for
safe parallel execution.

We have also shown how size-dependent array types can be inferred from
a size-agnostic source program, via a combination of type-driven rules
and function slicing.  Our slicing technique results in neglible
overhead in practise.

By introducing novel parallel streaming constructs, we provide better
support for efficient sequential execution of excess parallelism than
is possible with the classical parallel combinators.  We have also
shown how these constructs permit highly general fusion rules, in
particular permitting sequential fusion without losing the potential
for parallel execution.

All of these build up to our primary result, the moderate flattening
algorithm, which allows the efficient exploitation of easily
accessible ``common case'' parallelism.  The moderate flattening
algorithm sequentialises excess parallelism while keeping intact the
high-level invariants provided by the original parallel and purely
functional formulation, which permits further locality-of-reference
optimisations.  We have demonstrated this capability by showing how to
automatically repair some cases of non-coalesced memory accesses, as
well as performing simple block tiling of loops.  We argue that the
moderate flattening algorithm can be developed further into a
\textit{gradual flattening algorithm}, which uses multi-versioned code
to exploit a varying amount of parallelism of the program, dependent
on the characteristics of the input data encountered at runtime.

To support our claims, we have validated our approach on $21$
benchmark programs, which are compiled to GPU code via
OpenCL. Compared to reference implementations, the performance ranges
from $\times0.21$ slowdown to $\times13$ speedup, and is competitive
on average.
%
Our results show that while the ease of high-level structural
transformation permitted by a functional language is powerful,
attention must still be paid to low-level issues such as communication
costs and memory access patterns.

We have made the developed compiler and all benchmark programs freely
available for reproduction, study, or further development.

\section{Limitations and Future Work}

The primary limitation of Futhark as a language is the lack of support
for irregular arrays.  Some problems, such as graph algorithms, are
naturally irregular, and transforming them to a regular formulation is
tedious and error-prone.  Worse, such transformation is essentially
manual flattening, which tends to result in code that is hard for the
compiler to analyse and optimise.  It remains to be seen how we can
either modify the algorithms in question to exhibit less irregularity,
or extend Futhark with lightweight support for some cases of irregular
parallelism, without sacrificing the ability of the compiler to
generate efficient code.  After all, Futhark's entire \textit{raison
  d'être} is performance---there are already many functional languages
that are far more expressive, so improving flexibility at great cost
in runtime performance is not a goal.

However, even with the current language, irregularity still rears its
ugly head for the compiler.  It is not hard to write a program that,
although it uses only regular arrays, implicitly expresses irregular
\textit{parallelism}, which cannot in general be handled by our
current approach.  The reason is that we expect to be able to
pre-allocate memory before entering GPU kernels, but it is not hard to
write a program in which the size of intermediate arrays is
thread-variant.  For example, consider the following expression.
\begin{lstlisting}
map (\i -> reduce (+) 0 (iota i)) is
\end{lstlisting}
\noindent The size of the array produced by \lstinline{iota i} may
differ for each element of the array \texttt{is}, which means the
total memory requirement of the \kw{map} cannot be immediately known.
The problem has multiple solutions.  In the most general case, we can
apply full flattening---this removes all forms of irregularity, but
the overhead can be significant.  In other cases, a slicing approach
can be used.  For example, we can precompute the total memory
required, allocate one large slab of memory, and use a \kw{scan} to
compute an offset for each iteration of the \kw{map}:
\begin{lstlisting}
let os = scan (+) 0 is
in map (\(i,o) ->
          -- iota i located at offset 'o'
          -- in some memory block.
          reduce (+) 0 (iota i))
       (zip is os)
\end{lstlisting}

This is not a language issue---our sequential compiler pipeline is
able to compile any valid Futhark program to sequential code---but a
limitation in our compilation strategy for parallel code.  Our
strategy was to focus on development a technique for exploiting
``common case'' parallelism efficiently, and only later extend it to
support more exotic cases.

%%% Local Variables:
%%% mode: latex
%%% TeX-master: "thesis"
%%% End:
