\begin{abstract}

In this thesis we describe the design and implementation of Futhark, a
small data-parallel purely functional array language that offers a
machine-neutral programming model, and an optimising compiler that
generates efficient OpenCL code for GPUs.  The overall philosophy is
based on seeking a middle ground between functional and imperative
approaches.  The specific contributions are as follows:

First, we present a moderate flattening transformation aimed at
enhancing the degree of parallelism, which is capable of exploiting
easily accessible parallelism.  Excess parallelism is efficiently
sequentialised, while keeping access patterns intact, which then
permits further locality-of-reference optimisations.  We demonstrate
this capability by showing instances of automatic loop tiling, as well
as optimising memory access patterns.

Second, to support the flattening transformation, we present a
lightweight system of size-dependent types that enables the compiler
to reason symbolically about the size of arrays in the program, and
that reuses general-purpose compiler optimisations to infer
relationships between sizes.

Third, we furnish Futhark with novel parallel combinators capable of
expressing efficient sequentialisation of excess parallelism, as well
as their fusion rules.

Fourth, in order to express efficient programmer-written sequential
code inside parallel constructs, we introduce support for safe
in-place updates, with type system support to ensure referential
transparency and equational reasoning.

Fifth, we perform an evaluation on 21 benchmarks that demonstrates the
impact of the language and compiler features, and shows
application-level performance that is in many cases competitive with
hand-written GPU code.

Sixth, we make the Futhark compiler freely available with full source
code and documentation, to serve both as a basis for further
research into functional array programming, and as a useful tool
for parallel programming in practise.

\end{abstract}

\newpage

\renewcommand{\abstractname}{Resum\'e}

\begin{abstract}

  Denne afhandling beskriver udformningen og implementeringen af
  Futhark, et enkelt data-parallelt, sideeffekt-frit, og
  funktionsorienteret geledsprog, der frembyder en maskinneutral
  programmeringsmodel.  Vi beskriver ligeledes den tilhørende
  optimerende Futhark-oversætter, som producerer effektiv OpenCL-kode
  målrettet afvikling på GPUer.  Den overordnede designfilosofi er at
  ydnytte både functionsorienterede og imperative fremgangsmåder.
  Vores konkrete bidrag er som følger:

  For det første præsenterer vi en moderat fladningstransformering,
  der er i stand til at udnytte blot den grad af parallelisme som er
  nødvendig eller lettilgængelig, og omdanne overskydende parallelisme
  til effektiv sekventiel kode.  Denne sekventielle kode bibeholder
  oprindelig lageradgangsmønsterinformation, hvilket tillader
  yderligere lagertilgangsforbedringer.  Vi demonstrerer nytten af
  denne egenskab ved at give eksempler på automatisk blokafvikling af
  løkker, samt ændring af lageradgangsmønstre således at GPUens
  lagersystem udnyttes bedst muligt.

  For det andet beskriver vi, med henblik på understøttelse af
  fladningstransformeringen, et enkelt typesystem med
  størrelses-afhængige typer, der tillader oversætteren at ræsonnere
  symbolsk om størrelsen på geledder i programmet under oversættelse.
  Vores fremgangsmåde tillader genbrug af det almene repertoire af
  oversætteroptimeringer i spørgsmål om ligheder mellem størrelser.

  For det tredje udstyrer vi Futhark med en række nyskabedne
  parallelle kombinatorer der tillader effektiv sekventialisering af
  unødig parallelisme, samt disses fusionsregler.

  For det fjerde indfører vi, med henblik på at understøtte effektiv
  sekventiel kode indlejret i de parallelle sprogkonstruktioner,
  understøttelse for direkte ændringer i geledværdier.  Denne
  understøttelse sikres af et typesystem der garanterer at effekten
  ikke kan observeres, og at lighedsbaseret ræsonnering stadigvæk er
  muligt.

  For det femte foretager vi en ydelsessammenlining indeholdende 21
  programmer, med henblik på at demonstrere sprogets praktiske
  anvendelighed og oversætteroptimeringernes indvirkning.  Vores
  resultater viser at Futhark's overordnede ydelse i mange tilfælde er
  konkurrencedygtig med håndskreven GPU-kode.

  For det sjette gør vi Futhark-oversætteren frit tilgængelig,
  inklusive al kildetekst og omfattende dokumentation, således at den
  kan tjene både som et udgangspunkt for yderligere forskning i
  funktionsorienteret geledprogrammering, samt som et praktisk
  andvendeligt værktøj til parallelprogrammering.

\end{abstract}

%%% Local Variables:
%%% mode: latex
%%% TeX-master: "thesis"
%%% End:
