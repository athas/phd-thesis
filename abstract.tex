In this thesis we describe the design and implementation of Futhark, a
small data-parallel purely functional array language that offers a
machine-neutral programming model, and an optimising compiler that
generates efficient OpenCL code for GPUs.  The overall philosophy is
based on seeking a middle ground between functional and imperative
approaches.  The specific contributions are as follows:

First, we present a flattening transformation aimed at enhancing the
degree of parallelism that builds on loop interchange and distribution
but exploits properties directly derived from higher-order language
constructs rather than array-dependence analysis, while still
permitting further locality-of-reference optimisations.  We
demonstrate this capability by showing instances of automatic loop
tiling, as well as optimising memory access patterns.

Second, we present a lightweight system of size-dependent types that
enables the compiler to reason symbolically about the size of arrays
in the program, and that reuses general-purpose compiler optimisations
to infer relationships between sizes.

Third, we furnish Futhark with novel parallel operators capable of
expressing efficient sequentialisation of excess parallelism, as well
as their fusion rules.

Fourth, in order to express efficient programmer-written sequential
code inside parallel constructs, we introduce support for safe
in-place updates, with type system support to ensure referential
transparency and equational reasoning.

Fifth, we perform an evaluation on 21 benchmarks that demonstrates the
impact of the language and compiler features and shows
application-level performance that is in many cases competitive with
hand-written GPU code.

Sixth, we make the Futhark compiler freely available with full source
code and extensive documentation, to serve both as a basis for further
research into functional array programming, and as a useful tool
for parallel programming in practise.

%%% Local Variables:
%%% mode: latex
%%% TeX-master: "thesis"
%%% End:
