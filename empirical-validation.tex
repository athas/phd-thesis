\chapter{Empirical Validation in Bulk}
\label{chap:empirical-validation}

When a new programming model is proposed, it must be demonstrated that
interesting programs can be written within the model.  While the model
espoused by this thesis---functional array programming---is not new,
our claim that it is a good foundation for obtaining parallel
performance requires evidence.

This chapter contains a discussion of several programs implemented in
Futhark, with a focus on their runtime performance.  While we cannot
claim that the current Futhark compiler embodies the full potential of
functional array programming, the results presented here constitute at
least a lower bound on that potential.

Most of the benchmark programs presented here are based on, and
compared against, hand-written low-level code written in CUDA or
OpenCL.  We have sourced these from the published
Rodinia~\cite{5306797} (version 3.1),
Parboil~\cite{stratton2012parboil} (version 2.5), and
FinPar~\cite{FinPar:TACO} benchmark suites.  The comparison with
hand-written implementations is to show the cost of using a high-level
language.  However, as we shall see, even published code often
contains significant inefficiencies, which occasionally leads to
Futhark outperforming the reference implementations.  A handful of
benchmarks are from Accelerate~\cite{mcdonell2013optimising}, a
Haskell eDSL for parallel programming.  These are included to show the
performance of Futhark compared to an existing mature GPU language.

All programs have been manually ported to Futhark, and compiled and
run with the default settings.  We show how the performance of the
Futhark code compares to the performance of the original reference
implementations.  In some cases, we also discuss the programming
techniques used to obtain efficient Futhark implementations.

\section{N Benchmarks from Rodinia}

\section{M Benchmarks from Parboil}

\section{Two Benchmarks from FinPar}

\section{L Benchmarks from Accelerate}
\label{sec:accelerate}

%%% Local Variables:
%%% mode: latex
%%% TeX-master: "thesis"
%%% End:
